\documentclass[11pt]{exam}

%\printanswers

\usepackage{hyperref}
\usepackage{epsfig}
\usepackage{times}
\usepackage{latexsym}
\usepackage{amsmath}
\usepackage{amssymb}
\usepackage{graphics}
\newtheorem{problem}{Problem}

\setlength{\textwidth}{6.5in}
\setlength{\oddsidemargin}{-.2in}
\setlength{\textheight}{9in}
\setlength{\topmargin}{-1in}

\newcommand{\be}{\begin{enumerate}}
\newcommand{\ee}{\end{enumerate}}
\newcommand{\bc}{\begin{center}}
\newcommand{\ec}{\end{center}}
\newcommand{\bd}{\begin{description}}
\newcommand{\ed}{\end{description}}
\newcommand{\ba}{\begin{array}}
\newcommand{\ea}{\end{array}}
\newcommand{\beas}{\begin{eqnarray*}}
\newcommand{\eeas}{\end{eqnarray*}}
\newcommand{\bfr}{\begin{frame}}
\newcommand{\efr}{\end{frame}}
\newcommand{\bi}{\begin{itemize}}
\newcommand{\ei}{\end{itemize}}
\renewcommand{\P}{\boldsymbol{P}}
\newcommand{\Y}{\boldsymbol{Y}}
\newcommand{\y}{\boldsymbol{y}}
\newcommand{\yh}{\widehat{\y}}
\newcommand{\z}{\boldsymbol{z}}
\newcommand{\X}{\boldsymbol{X}}
\newcommand{\x}{\boldsymbol{x}}
\newcommand{\A}{\boldsymbol{A}}
\newcommand{\B}{\boldsymbol{B}}
\renewcommand{\a}{\boldsymbol{a}}
\newcommand{\C}{\boldsymbol{C}}
\newcommand{\I}{\boldsymbol{I}}
\newcommand{\1}{\boldsymbol{1}}
\newcommand{\0}{\boldsymbol{0}}
\newcommand{\proj}{\P_{\X}}
\def\d{\boldsymbol{d}}
\newcommand{\e}{\boldsymbol{\epsilon}}
\renewcommand{\b}{\boldsymbol{\beta}}
\newcommand{\bmu}{\boldsymbol{\mu}}
\newcommand{\bSigma}{\boldsymbol{\Sigma}}
\newcommand{\Prob}{I\!P}
\newcommand{\ind}{1\!\!1}
\newcommand{\R}{I\!\!R}
\newcommand{\cov}{\rm Cov}
\newcommand{\var}{\rm Var}
\newcommand{\E}{\rm E}
\newcommand{\cs}{\mathcal C}
\newcommand{\bb}{\boldsymbol{b}}
\newcommand{\tr}{\rm tr}
\newcommand{\diag}{\rm diag}
\def\c{\boldsymbol{c}}


%some new commands
\newcommand{\bs}{\begin{solution}}
\newcommand{\es}{\end{solution}}
\newcommand{\lp}{\left(}
\newcommand{\rp}{\right)}
\newcommand{\D}{\boldsymbol{D}}
\newcommand{\F}{\boldsymbol{F}}
\newcommand{\U}{\boldsymbol{U}}
\newcommand{\G}{\boldsymbol{G}}
\newcommand{\W}{\boldsymbol{W}}
\renewcommand{\W}{\boldsymbol{W}}
\renewcommand{\H}{\boldsymbol{H}}
\newcommand{\balpha}{\boldsymbol{\alpha}}
%%%%%%%%%%%%%%%%%%%%%

\begin{document}

\noindent {\large \bf Stat 5100 
\hfill  {\large Assignment 5 }} \\


\noindent {\bf Due:} Wednesday, February 19th 11:59PM in gradescope. \\
{\bf Late Due Date without penalty:}  Friday February 28 11:59PM in gradescope. \\

\hrulefill

\begin{problem}{\textbf{Cell means versus Additive Model.}}
In class, we talked extensively about the two types of models when analyzing two treatment factors. For this question, consider your audience to be a first year graduate student in	 a different discipline. Their understanding of statistics includes what you learned in Stat 5000 but they have not seen much Stat 5100 materials. The student is asking you for help trying to better understand 

\begin{enumerate}
	\item[a)] the two types of statistical models themselves. \\[5cm]
	\item[b)] the difference between both types of statistical models.  \\[5cm]
	\item[c)] which one they should use for their own experiment that they plan on carrying out studying the effect of two treatments on some response $y$.\\[5cm]
\end{enumerate}


\end{problem}



\begin{problem} {\textbf{The surprising power of reflection.}} What can reflection, purposeful reflection do for our learning?

\begin{enumerate}
\item[a)] Please watch the following video (6 min 43 seconds): 

\href{https://thelearnerlab.us4.list-manage.com/track/click?u=650effaf591ee7171e0472541&id=ba9ae9d9c6&e=552083b364}{The surprising power of reflection.} You should be able to click on the link but I will also post the link in Canvas with the assignment.
\item[b)] The video showcases two studies: one immediately in the beginning within the first minute of the video and a second one introduced in the last 45 seconds of minute two.  Listen carefully to the descriptions of each study. What \textit{flaw}, statistically speaking, does the first study suffer from that does not show up in the second study. Briefly explain.\\ \\[4cm]
\item[c)] In 2012, I took a workshop at ISU with Dr.\ Jan Wiersema called Project LEA/RN. Dr.\ Wiersema shared the following advice with us during the workshop -- it has stuck with me ever since: 
\begin{center}
\textbf{It's the thinking about the doing that does the learning.}
\end{center}
Reflect on, and briefly summarize how \textbf{you} best learn new things. Share one or two tips on how you deal with challenging course material to ensure you learn it. You can reference experiences you have made as a student since joining our program but you can also reference an experience at some other time in your life. \\
Be sure your response has a reflection piece and a tip or two in addition. I plan on sharing your tips (anonymously!).\\
\end{enumerate}
\end{problem}




\newpage
\begin{problem}  Go to the old \textbf{MS exam repository} and look at the Methods I and II questions. Familiarize yourself with the questions in Methods I -- you should have an idea on how to answer most questions that are part of Methods I. Familiarize yourself with the questions in Methods II -- these you cannot answer yet, for the most part. Select one Methods II question you find intriguing and download the question document. \\
    
    \underline{If you are in the PhD program in Statistics}, in addition, pick and download a Methods II question from the old \textbf{PhD exam repository.}
    
 \begin{enumerate}   
    
\item[a)] Which questions did you pick? Answer by following this format: YEAR Methods II MS repository. (and YEAR Methods II PhD repository if you are a PhD student). \\[2cm]

\item[b)] Submit your question(s) as part of the homework. Note that I am not asking you to solve the question(s) (\textbf{yet}); I just want you to familiarize with them so you have an idea about expectations. For the new PhD qualifying exam, expect the level of difficulty to fall approximately in between old MS exam and old PhD exam questions. \\

Submit at the end of your document.
\item[c)] Reflect on your Fall semester; what has helped you learn and why? Note that it is the \textit{why-part} of your answer that I am most interested in. 
\end{enumerate}
\end{problem}
\vfill

\textbf{References:}


\begin{itemize}
\item Di Stefano, Giada and Gino, Francesca and Pisano, Gary and Staats, Bradley R., Learning by Thinking: How Reflection Can Spur Progress Along the Learning Curve (February 6, 2023). Harvard Business School NOM Unit Working Paper No. 14-093, Kenan Institute of Private Enterprise Research Paper No. 2414478, Available at \href{https://dx.doi.org/10.2139/ssrn.2414478}{https://dx.doi.org/10.2139/ssrn.2414478}
\item Link: \href{https://papers.ssrn.com/sol3/papers.cfm?abstract_id=2414478}{Learning by Thinking: How Reflection Can Spur Progress Along the Learning Curve}
\end{itemize}

\end{document}
