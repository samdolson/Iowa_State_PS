% Options for packages loaded elsewhere
\PassOptionsToPackage{unicode}{hyperref}
\PassOptionsToPackage{hyphens}{url}
\documentclass[
  ignorenonframetext,
]{beamer}
\newif\ifbibliography
\usepackage{pgfpages}
\setbeamertemplate{caption}[numbered]
\setbeamertemplate{caption label separator}{: }
\setbeamercolor{caption name}{fg=normal text.fg}
\beamertemplatenavigationsymbolsempty
% remove section numbering
\setbeamertemplate{part page}{
  \centering
  \begin{beamercolorbox}[sep=16pt,center]{part title}
    \usebeamerfont{part title}\insertpart\par
  \end{beamercolorbox}
}
\setbeamertemplate{section page}{
  \centering
  \begin{beamercolorbox}[sep=12pt,center]{section title}
    \usebeamerfont{section title}\insertsection\par
  \end{beamercolorbox}
}
\setbeamertemplate{subsection page}{
  \centering
  \begin{beamercolorbox}[sep=8pt,center]{subsection title}
    \usebeamerfont{subsection title}\insertsubsection\par
  \end{beamercolorbox}
}
% Prevent slide breaks in the middle of a paragraph
\widowpenalties 1 10000
\raggedbottom
\AtBeginPart{
  \frame{\partpage}
}
\AtBeginSection{
  \ifbibliography
  \else
    \frame{\sectionpage}
  \fi
}
\AtBeginSubsection{
  \frame{\subsectionpage}
}
\usepackage{iftex}
\ifPDFTeX
  \usepackage[T1]{fontenc}
  \usepackage[utf8]{inputenc}
  \usepackage{textcomp} % provide euro and other symbols
\else % if luatex or xetex
  \usepackage{unicode-math} % this also loads fontspec
  \defaultfontfeatures{Scale=MatchLowercase}
  \defaultfontfeatures[\rmfamily]{Ligatures=TeX,Scale=1}
\fi
\usepackage{lmodern}
\usetheme[]{Madrid}
\usefonttheme[]{professionalfonts}
\ifPDFTeX\else
  % xetex/luatex font selection
\fi
% Use upquote if available, for straight quotes in verbatim environments
\IfFileExists{upquote.sty}{\usepackage{upquote}}{}
\IfFileExists{microtype.sty}{% use microtype if available
  \usepackage[]{microtype}
  \UseMicrotypeSet[protrusion]{basicmath} % disable protrusion for tt fonts
}{}
\makeatletter
\@ifundefined{KOMAClassName}{% if non-KOMA class
  \IfFileExists{parskip.sty}{%
    \usepackage{parskip}
  }{% else
    \setlength{\parindent}{0pt}
    \setlength{\parskip}{6pt plus 2pt minus 1pt}}
}{% if KOMA class
  \KOMAoptions{parskip=half}}
\makeatother
% definitions for citeproc citations
\NewDocumentCommand\citeproctext{}{}
\NewDocumentCommand\citeproc{mm}{%
  \begingroup\def\citeproctext{#2}\cite{#1}\endgroup}
\makeatletter
 % allow citations to break across lines
 \let\@cite@ofmt\@firstofone
 % avoid brackets around text for \cite:
 \def\@biblabel#1{}
 \def\@cite#1#2{{#1\if@tempswa , #2\fi}}
\makeatother
\newlength{\cslhangindent}
\setlength{\cslhangindent}{1.5em}
\newlength{\csllabelwidth}
\setlength{\csllabelwidth}{3em}
\newenvironment{CSLReferences}[2] % #1 hanging-indent, #2 entry-spacing
 {\begin{list}{}{%
  \setlength{\itemindent}{0pt}
  \setlength{\leftmargin}{0pt}
  \setlength{\parsep}{0pt}
  % turn on hanging indent if param 1 is 1
  \ifodd #1
   \setlength{\leftmargin}{\cslhangindent}
   \setlength{\itemindent}{-1\cslhangindent}
  \fi
  % set entry spacing
  \setlength{\itemsep}{#2\baselineskip}}}
 {\end{list}}
\usepackage{calc}
\newcommand{\CSLBlock}[1]{\hfill\break\parbox[t]{\linewidth}{\strut\ignorespaces#1\strut}}
\newcommand{\CSLLeftMargin}[1]{\parbox[t]{\csllabelwidth}{\strut#1\strut}}
\newcommand{\CSLRightInline}[1]{\parbox[t]{\linewidth - \csllabelwidth}{\strut#1\strut}}
\newcommand{\CSLIndent}[1]{\hspace{\cslhangindent}#1}
\setlength{\emergencystretch}{3em} % prevent overfull lines
\providecommand{\tightlist}{%
  \setlength{\itemsep}{0pt}\setlength{\parskip}{0pt}}
\usepackage{xurl}          % better URL/DOI wrapping
\usepackage{microtype}     % nicer spacing
\setbeamertemplate{bibliography item}[text]  % no bullets on refs
\usepackage{bookmark}
\IfFileExists{xurl.sty}{\usepackage{xurl}}{} % add URL line breaks if available
\urlstyle{same}
\hypersetup{
  pdftitle={NP Report},
  pdfauthor={Sam Olson},
  hidelinks,
  pdfcreator={LaTeX via pandoc}}

\title{NP Report}
\author{Sam Olson}
\date{2025-09-29}

\begin{document}
\frame{\titlepage}

\begin{frame}{General Timeline}
\phantomsection\label{general-timeline}
\begin{itemize}
\tightlist
\item
  \textbf{1897} --- Fechner introduces the \emph{method of signs} for
  succession-dependence.
\item
  \textbf{1938} --- Kendall develops the \(\tau\) rank correlation
  coefficient.
\item
  \textbf{1958} --- Kruskal broadens Kendall's ideas into a general
  nonparametric testing framework.
\item
  \textbf{1958--1990s} --- Others (e.g., El-Shaarawi, 1992) apply
  rank-based methods to time series.
\item
  \textbf{2024} --- Shi et al.~develop adaptive high-dimensional
  independence tests using Kendall's \(\tau\).
\item
  \textbf{2025} --- Han et al.~extend to a broader class of
  sum-of-powers tests.
\end{itemize}
\end{frame}

\begin{frame}[allowframebreaks]{General Summary}
\phantomsection\label{general-summary}
The trajectory from Fechner to modern adaptive tests highlights how a
simple sign-based idea grew into a major branch of nonparametric
inference.

Fechner's Kollektivmasslehre (1897) (Fechner 1897) anticipated many of
the ideas behind Kendall's \(\tau\). His method of signs looked for
succession-dependence in sequences of observations, asking whether runs
of increases or decreases occurred more often than chance would predict.
Though Fechner restricted his comparisons to adjacent pairs, the spirit
was the same: assess concordance and discordance using only the signs of
differences, not their magnitudes. He even applied this method to
meteorological series and anthropometric data, extending the idea to two
dimensions.

Kendall (1938) (Kendall 1938) generalized Fechner's idea by considering
all possible pairs of observations, not just consecutive ones. His
\(\tau\) statistic became the canonical rank correlation coefficient,
widely adopted as a nonparametric alternative to Pearson's correlation.

Kruskal (1958) (Kruskal 1958) emphasized \(\tau\)'s place within a
broader family of nonparametric statistics for ordinal data, framing it
for hypothesis testing. Rank-based measures then spread to time series,
enabling tests for persistence/independence in hydrological and
environmental data (El-Shaarawi and Niculescu 1992; Hamed 2011).

Key point: This lineage leads to independence testing in high
dimensions, where Kendall's \(\tau\) supports robust, distribution-free
procedures resilient to heavy tails and monotone transformations (Shi et
al. 2024; Han, Ma, and Xie 2025).
\end{frame}

\begin{frame}{Concise Comparison to Kendall's \(\tau\)}
\phantomsection\label{concise-comparison-to-kendalls-tau}
Fechner (1897): Successive changes in time series; sign-based
concordance on adjacent pairs.

Kendall (1938): Concordance/discordance over all pairs; a general rank
correlation for unordered data (Kendall 1938).

Key takeaway: Fechner's succession-based idea becomes Kendall's general
ordinal association measure.
\end{frame}

\begin{frame}{Motivation and Relevance}
\phantomsection\label{motivation-and-relevance}
Modern work continues to exploit distribution-free, rank-based tests of
independence:

Adaptive high-dimensional tests building on Kendall's \(\tau\) (Shi et
al. 2024; Han, Ma, and Xie 2025).

Time-series applications echoing Fechner's focus (El-Shaarawi and
Niculescu 1992).

Broader treatments of ordinal association and nonparametric effects
(Kruskal 1958; Newson, n.d.).

Persistence testing with ranks in environmental contexts (Hamed 2011).
\end{frame}

\begin{frame}[allowframebreaks]{Recent Development: Shi et al.~(2024)}
\phantomsection\label{recent-development-shi-et-al.-2024}
\begin{block}{Problem}
\phantomsection\label{problem}
\[
H_0:; X_1,\dots,X_d \text{ are mutually independent.}
\]
\end{block}

\begin{block}{Why Kendall's \(\tau\)?}
\phantomsection\label{why-kendalls-tau}
\begin{itemize}
\tightlist
\item
  Rank-based; distribution-free; robust to heavy tails.
\end{itemize}
\end{block}

\begin{block}{Dense vs.~Sparse}
\phantomsection\label{dense-vs.-sparse}
\begin{itemize}
\tightlist
\item
  \textbf{Dense:} many weak deps \(\Rightarrow\) sum-type (\(L_2\)).
\item
  \textbf{Sparse:} few strong deps \(\Rightarrow\) max-type
  (\(L_\infty\)).
\end{itemize}
\end{block}

\begin{block}{Method (sketch)}
\phantomsection\label{method-sketch}
\begin{itemize}
\tightlist
\item
  Build \(L_2\) and \(L_\infty\) from pairwise \(\tau_{k\ell}\).
\item
  \(S_\tau \Rightarrow N(0,1)\); \(M_\tau \Rightarrow\) Gumbel.
\item
  Adaptive p-value:
\end{itemize}

\[
C_\tau=\min{(1-\Phi(S_\tau), 1-F_{\mathrm{Gumbel}}(M_\tau))}
\]
\end{block}

\begin{block}{Theory (high level)}
\phantomsection\label{theory-high-level}
\(S_\tau\) and \(M_\tau\) asymptotically independent;
\(W=\min{U_1,U_2}\) with \(U_i\sim \mathrm{Unif}(0,1)\) so
\(H(t)=2t-t^2\).
\end{block}

\begin{block}{Empirics / Applications}
\phantomsection\label{empirics-applications}
\begin{itemize}
\tightlist
\item
  \textbf{Welding (4 vars, n=40):} rank-based rejects; Pearson fails.
\item
  \textbf{Biochemical (8 vars):} adaptive detects group differences.
\end{itemize}
\end{block}
\end{frame}

\begin{frame}{Conclusion}
\phantomsection\label{conclusion}
Rank-based adaptive tests are practical and robust; 2025 work
generalizes to sum-of-powers (Han, Ma, and Xie 2025).
\end{frame}

\begin{frame}{Next Steps}
\phantomsection\label{next-steps}
\begin{itemize}
\tightlist
\item
  Consulting applications (survey, environmental, biochemical).
\item
  Reflection: Kendall's \(\tau\) connects classic nonparametrics to
  modern HD inference.
\end{itemize}
\end{frame}

\begin{frame}[allowframebreaks]{References}
\phantomsection\label{references}
\scriptsize

\phantomsection\label{refs}
\begin{CSLReferences}{1}{0}
\bibitem[\citeproctext]{ref-ElShaarawi1992}
El-Shaarawi, A. H., and Stefan P. Niculescu. 1992. {``On Kendall's Tau
as a Test of Trend in Time Series Data.''} \emph{Environmetrics} 3 (4):
385--411.

\bibitem[\citeproctext]{ref-Fechner1897}
Fechner, Gustav Theodor. 1897. \emph{Kollektivmasslehre}. Leipzig:
Verlag von Wilhelm Engelmann.
\url{https://www.google.com/books/edition/Kollektivmasslehre/bgQZAAAAMAAJ?hl=en}.

\bibitem[\citeproctext]{ref-Hamed2011}
Hamed, K. H. 2011. {``The Distribution of Kendall's Tau for Testing the
Significance of Cross-Correlation in Persistent Data.''}
\emph{Hydrological Sciences Journal} 56 (5): 841--53.
\url{https://doi.org/10.1080/02626667.2011.586948}.

\bibitem[\citeproctext]{ref-Han2025}
Han, Lijuan, Yun Ma, and Junshan Xie. 2025. {``An Adaptive Test of the
Independence of High-Dimensional Data Based on Kendall Rank Correlation
Coefficient.''} \emph{Journal of Nonparametric Statistics} 37 (3):
632--56. \url{https://doi.org/10.1080/10485252.2024.2435852}.

\bibitem[\citeproctext]{ref-Kendall1938}
Kendall, M. G. 1938. {``A New Measure of Rank Correlation.''}
\emph{Biometrika} 30 (1/2): 81--93.
\url{https://www.jstor.org/stable/2332226}.

\bibitem[\citeproctext]{ref-Kruskal1958}
Kruskal, William H. 1958. {``Ordinal Measures of Association.''}
\emph{Journal of the American Statistical Association} 53 (284):
814--61. \url{https://doi.org/10.1080/01621459.1958.10501481}.

\bibitem[\citeproctext]{ref-Newson}
Newson, Roger. n.d. {``Parameters Behind {`Nonparametric'} Statistics:
Kendall's Tau, Somers' d and Median Differences.''} Working paper,
King's College London.

\bibitem[\citeproctext]{ref-Shi2024}
Shi, Xiangyu, Yuanyuan Jiang, Jiang Du, and Zhuqing Miao. 2024. {``An
Adaptive Test Based on Kendall's Tau for Independence in High
Dimensions.''} \emph{Journal of Nonparametric Statistics} 36 (4):
1064--87. \url{https://doi.org/10.1080/10485252.2023.2296521}.

\end{CSLReferences}
\end{frame}

\end{document}
