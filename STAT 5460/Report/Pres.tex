% Options for packages loaded elsewhere
\PassOptionsToPackage{unicode}{hyperref}
\PassOptionsToPackage{hyphens}{url}
\documentclass[
  ignorenonframetext,
]{beamer}
\newif\ifbibliography
\usepackage{pgfpages}
\setbeamertemplate{caption}[numbered]
\setbeamertemplate{caption label separator}{: }
\setbeamercolor{caption name}{fg=normal text.fg}
\beamertemplatenavigationsymbolsempty
% remove section numbering
\setbeamertemplate{part page}{
  \centering
  \begin{beamercolorbox}[sep=16pt,center]{part title}
    \usebeamerfont{part title}\insertpart\par
  \end{beamercolorbox}
}
\setbeamertemplate{section page}{
  \centering
  \begin{beamercolorbox}[sep=12pt,center]{section title}
    \usebeamerfont{section title}\insertsection\par
  \end{beamercolorbox}
}
\setbeamertemplate{subsection page}{
  \centering
  \begin{beamercolorbox}[sep=8pt,center]{subsection title}
    \usebeamerfont{subsection title}\insertsubsection\par
  \end{beamercolorbox}
}
% Prevent slide breaks in the middle of a paragraph
\widowpenalties 1 10000
\raggedbottom
\AtBeginPart{
  \frame{\partpage}
}
\AtBeginSection{
  \ifbibliography
  \else
    \frame{\sectionpage}
  \fi
}
\AtBeginSubsection{
  \frame{\subsectionpage}
}
\usepackage{iftex}
\ifPDFTeX
  \usepackage[T1]{fontenc}
  \usepackage[utf8]{inputenc}
  \usepackage{textcomp} % provide euro and other symbols
\else % if luatex or xetex
  \usepackage{unicode-math} % this also loads fontspec
  \defaultfontfeatures{Scale=MatchLowercase}
  \defaultfontfeatures[\rmfamily]{Ligatures=TeX,Scale=1}
\fi
\usepackage{lmodern}
\usetheme[]{Madrid}
\usefonttheme[]{professionalfonts}
\ifPDFTeX\else
  % xetex/luatex font selection
\fi
% Use upquote if available, for straight quotes in verbatim environments
\IfFileExists{upquote.sty}{\usepackage{upquote}}{}
\IfFileExists{microtype.sty}{% use microtype if available
  \usepackage[]{microtype}
  \UseMicrotypeSet[protrusion]{basicmath} % disable protrusion for tt fonts
}{}
\makeatletter
\@ifundefined{KOMAClassName}{% if non-KOMA class
  \IfFileExists{parskip.sty}{%
    \usepackage{parskip}
  }{% else
    \setlength{\parindent}{0pt}
    \setlength{\parskip}{6pt plus 2pt minus 1pt}}
}{% if KOMA class
  \KOMAoptions{parskip=half}}
\makeatother
\usepackage{longtable,booktabs,array}
\usepackage{calc} % for calculating minipage widths
\usepackage{caption}
% Make caption package work with longtable
\makeatletter
\def\fnum@table{\tablename~\thetable}
\makeatother
% definitions for citeproc citations
\NewDocumentCommand\citeproctext{}{}
\NewDocumentCommand\citeproc{mm}{%
  \begingroup\def\citeproctext{#2}\cite{#1}\endgroup}
\makeatletter
 % allow citations to break across lines
 \let\@cite@ofmt\@firstofone
 % avoid brackets around text for \cite:
 \def\@biblabel#1{}
 \def\@cite#1#2{{#1\if@tempswa , #2\fi}}
\makeatother
\newlength{\cslhangindent}
\setlength{\cslhangindent}{1.5em}
\newlength{\csllabelwidth}
\setlength{\csllabelwidth}{3em}
\newenvironment{CSLReferences}[2] % #1 hanging-indent, #2 entry-spacing
 {\begin{list}{}{%
  \setlength{\itemindent}{0pt}
  \setlength{\leftmargin}{0pt}
  \setlength{\parsep}{0pt}
  % turn on hanging indent if param 1 is 1
  \ifodd #1
   \setlength{\leftmargin}{\cslhangindent}
   \setlength{\itemindent}{-1\cslhangindent}
  \fi
  % set entry spacing
  \setlength{\itemsep}{#2\baselineskip}}}
 {\end{list}}
\usepackage{calc}
\newcommand{\CSLBlock}[1]{\hfill\break\parbox[t]{\linewidth}{\strut\ignorespaces#1\strut}}
\newcommand{\CSLLeftMargin}[1]{\parbox[t]{\csllabelwidth}{\strut#1\strut}}
\newcommand{\CSLRightInline}[1]{\parbox[t]{\linewidth - \csllabelwidth}{\strut#1\strut}}
\newcommand{\CSLIndent}[1]{\hspace{\cslhangindent}#1}
\setlength{\emergencystretch}{3em} % prevent overfull lines
\providecommand{\tightlist}{%
  \setlength{\itemsep}{0pt}\setlength{\parskip}{0pt}}
\usepackage{xurl}          % better URL/DOI wrapping
\usepackage{microtype}     % nicer spacing
\setbeamertemplate{bibliography item}[text]  % no bullets on refs
\usepackage{bookmark}
\IfFileExists{xurl.sty}{\usepackage{xurl}}{} % add URL line breaks if available
\urlstyle{same}
\hypersetup{
  pdftitle={An adaptive test based on Kendall's \textbackslash tau},
  pdfauthor={Sam Olson},
  hidelinks,
  pdfcreator={LaTeX via pandoc}}

\title{An adaptive test based on Kendall's \(\tau\)}
\subtitle{for independence in high dimensions (Shi et al. 2024)}
\author{Sam Olson}
\date{}

\begin{document}
\frame{\titlepage}

\begin{frame}{Synopsis of Paper}
\phantomsection\label{synopsis-of-paper}
\begin{itemize}
\tightlist
\item
  Testing complete (mutual) independence in high-dimensional data
  (\(d \gg n\))
\item
  Known that \(L_2\)-type statistics have lower power under sparse cases
\item
  Known that \(L_\infty\)-type statistics have lower power under dense
  cases
\item
  \textbf{Goal:} Develop an adaptive test based on Kendall's \(\tau\) to
  work well in both situations
\item
  Determine necessary assumptions and the asymptotic null distribution
  of the proposed statistic
\item
  Assess how well the test does compared to other testing methods
\item
  Results indicate the adaptive test performs well in either dense or
  sparse cases
\end{itemize}
\end{frame}

\begin{frame}{What is Kendall's \(\tau\)?}
\phantomsection\label{what-is-kendalls-tau}
\[
\tau = \frac{(\text{Count of concordant pairs}) - (\text{Count of discordant pairs})}{(\text{Number of pairs})}
\]

\begin{itemize}
\tightlist
\item
  Any pair of observations \((x_i, y_i)\) and \((x_j, y_j)\), where
  \(i < j\), are said to be \emph{concordant} if the sort order of
  \((x_i, x_j)\) and \((y_i, y_j)\) agrees.
\item
  That is, if either both \(x_i > x_j\) and \(y_i > y_j\) holds or both
  \(x_i < x_j\) and \(y_i < y_j\); otherwise they are said to be
  \emph{discordant}.
\item
  There are other types of Kendall's \(\tau\), e.g., \(\tau_a\),
  \(\tau_b\), and \(\tau_c\).
\end{itemize}
\end{frame}

\begin{frame}{Kendall's \(\tau\) in (Shi et al. 2024)}
\phantomsection\label{kendalls-tau-in-shi2024}
In the paper, Kendall's \(\tau\) between \(X_k\) and \(X_\ell\) is
defined as

\[
\tau_{k\ell}
  = \frac{2}{n(n-1)}
    \sum_{i=2}^n \sum_{j=1}^{i-1}
    \operatorname{sign}(R_{ki}-R_{kj})\,
    \operatorname{sign}(R_{\ell i}-R_{\ell j}),
\]

where \(R_{ki}\) and \(R_{\ell i}\) are the ranks of \(X_k\) and
\(X_\ell\).

This is equivalent to the definition given on the prior slide.
\end{frame}

\begin{frame}{History of Kendall's \(\tau\) and related concepts}
\phantomsection\label{history-of-kendalls-tau-and-related-concepts}
\begin{itemize}
\tightlist
\item
  \textbf{1897} --- (Fechner 1897) introduces the \emph{method of signs}
  for succession-dependence.
\item
  \textbf{1938} --- (Kendall 1938) develops the \(\tau\) rank
  correlation coefficient.
\item
  \textbf{1958} --- (Kruskal 1958) generalizes Kendall's ideas into a
  general nonparametric testing framework.
\item
  \textbf{1958--1990s} --- Applied to time series settings for tests of
  serial dependence. (El-Shaarawi and Niculescu 1992; Hamed 2011).
\item
  \textbf{2020-2024} --- Used to determine whether two separate
  processes replicate metrics (application in healthcare ``match rate''
  analysis; personal experience).
\item
  \textbf{2024} --- (Shi et al. 2024), \textbf{the focus of this
  presentation}, develop adaptive high-dimensional independence tests
  using Kendall's \(\tau\).
\item
  \textbf{2025} --- (Han, Ma, and Xie 2025) extend to a broader class of
  sum-of-powers tests.
\end{itemize}
\end{frame}

\begin{frame}{In Detail: Shi et al.~(2024): Problem Statement}
\phantomsection\label{in-detail-shi-et-al.-2024-problem-statement}
Let \(X = (X_1, \ldots, X_d)\) be a continuous random vector, with
i.i.d. observations \(x_i\).

\[
H_0: X_1, \dots, X_d \text{ are mutually independent}
\]

Testing full independence in \textbf{high dimensions} (\(d \gg n\)),
where both \emph{dense} and \emph{sparse} alternatives may occur.
\end{frame}

\begin{frame}{Why Kendall's \(\tau\)?}
\phantomsection\label{why-kendalls-tau}
\begin{itemize}
\tightlist
\item
  Rank-based, distribution-free, and robust to heavy tails.\\
\item
  Each \(\tau_{k\ell}\) measures pairwise monotonic dependence between
  \(X_k, X_\ell\).
\item
  Works even when moments (e.g., variances) are infinite.\\
\item
  Avoids dependence on the data-generation process (no parametric
  assumptions!)\\
\item
  Only requires continuous marginals to avoid ties.
\end{itemize}
\end{frame}

\begin{frame}{Dense vs.~Sparse Settings}
\phantomsection\label{dense-vs.-sparse-settings}
\begin{longtable}[]{@{}lll@{}}
\toprule\noalign{}
Setting & Dependence Structure & Suitable Statistic \\
\midrule\noalign{}
\endhead
\textbf{Dense} & Many weak correlations & \(L_2\)-type (sum-type) \\
\textbf{Sparse} & Few strong correlations & \(L_\infty\)-type
(max-type) \\
\bottomrule\noalign{}
\end{longtable}
\end{frame}

\begin{frame}{Method (Overview) I}
\phantomsection\label{method-overview-i}
\begin{enumerate}
\tightlist
\item
  Compute pairwise Kendall's taus \(\tau_{k\ell}\).\\
\item
  Construct two base statistics:
\end{enumerate}

\begin{itemize}
\tightlist
\item
  \(S_\tau\) (\(L_2\)-type):
\end{itemize}

\[
S_\tau = \omega_2^{-1/2}\left(\sum_{k>\ell}\tau_{k\ell}^2 - \frac{d(d-1)}{2}\,\omega_1\right),
\qquad
S_\tau \xrightarrow{d} N(0,1).
\]
\end{frame}

\begin{frame}{Method (Overview) II}
\phantomsection\label{method-overview-ii}
\begin{itemize}
\tightlist
\item
  \(M_\tau\) (\(L_\infty\)-type):
\end{itemize}

\[
M_{\tau}
= \omega_1^{-1}\!\left( \max_{k < \ell} \tau_{k\ell}^2 \right)
\;-\; 4 \ln d \;+\; \ln \ln d, 
\qquad
M_\tau \xrightarrow{d} \text{Gumbel}.
\]

where \(\omega_1, \omega_2\) are constants reflecting the variance
structure of pairwise Kendall's \(\tau\) under independence (\(H_0\)),
specifically:

\[
\omega_1 = \frac{2(2n+5)}{9n(n-1)}, \qquad
\omega_2 = \frac{4d(d-1)(n-2)(100n^3 + 492n^2 + 731n + 279)}{2025\,n^3(n-1)^3}
\]

Note: The use of ``ln ln'' is correct, both in the original paper and
the paper it cites.
\end{frame}

\begin{frame}{Method (Overview) III}
\phantomsection\label{method-overview-iii}
\begin{enumerate}
\setcounter{enumi}{2}
\tightlist
\item
  Combine the two via the \emph{minimum p-value approach} (an ``adaptive
  test''):
\end{enumerate}

\[
C_\tau = \min\{1 - F(M_\tau),\, 1 - \Phi(S_\tau)\},
\]

where \(\Phi\) is the standard normal CDF and \(F\) is the Gumbel CDF.
\end{frame}

\begin{frame}{A Quick Aside}
\phantomsection\label{a-quick-aside}
What is an Adaptive Test?

\begin{itemize}
\tightlist
\item
  A single procedure that automatically adapts to the dependence
  pattern.\\
\item
  If data are dense, \(S_\tau\) dominates; if sparse, \(M_\tau\)
  dominates.\\
\item
  \(C_\tau\) effectively selects the stronger signal through
  \(p\text{-value} = \min(p_{S_\tau}, p_{M_\tau})\).
\end{itemize}
\end{frame}

\begin{frame}{Theoretical Results I}
\phantomsection\label{theoretical-results-i}
With the following base assumptions:

\begin{itemize}
\tightlist
\item
  \(X = (X_1,\dots,X_d)\) has continuous marginals.
\item
  Observations \(x_{\cdot,i}\) are i.i.d.
\item
  \(\ln d = o(n^{1/3})\) as \(n \to \infty\).
\end{itemize}

Then, under \(H_0\):

\begin{itemize}
\tightlist
\item
  \(S_\tau\) and \(M_\tau\) are \textbf{asymptotically independent}
\item
  Therefore,
\end{itemize}

\[
(S_\tau, M_\tau) \xrightarrow{d} (Z_1, Z_2)
\quad \text{with } Z_1\sim N(0,1),\; Z_2\sim \text{Gumbel}.
\]
\end{frame}

\begin{frame}{Theoretical Results II}
\phantomsection\label{theoretical-results-ii}
Consequently, under \(H_0\):

\[
C_\tau=\min\{1-\Phi(S_\tau),\,1-F(M_\tau)\}
\, \xrightarrow{d} \, 
W=\min(U_1,U_2)
\]

where \(U_1,U_2\sim\text{Unif}(0,1)\)

The limiting CDF of \(W\) is

\[
H(t)=\Pr(W\le t)=2t - t^2,\qquad t\in[0,1]
\]

\textbf{Decision rule:} Reject \(H_0\) if

\[
C_\tau < 1-\sqrt{\,1-\alpha\,}
\]
\end{frame}

\begin{frame}{Practical Implementation}
\phantomsection\label{practical-implementation}
\begin{itemize}
\tightlist
\item
  Two variants:

  \begin{itemize}
  \tightlist
  \item
    \(TC_\tau\): Uses theoretical (asymptotic) critical values.\\
  \item
    \(MC_\tau\): Uses Monte Carlo--simulated critical values
    (finite-sample accurate).\\
  \end{itemize}
\item
  Distribution-free; efficient table lookup possible for \((n,d)\).
\end{itemize}
\end{frame}

\begin{frame}{Key Properties of the Adaptive Test}
\phantomsection\label{key-properties-of-the-adaptive-test}
\begin{itemize}
\tightlist
\item
  Adaptive: Unified test for both dense and sparse dependence.\\
\item
  Joint Asymptotic independence of \(S_{\tau}, M_{\tau}\) allows for an
  adaptive combination (hence the valid statistical test of
  independence)
\item
  Asymptotic theory:

  \begin{itemize}
  \tightlist
  \item
    \(S_\tau \to N(0,1)\)\\
  \item
    \(M_\tau \to \text{Gumbel}\)\\
  \item
    \(S_\tau, M_\tau\) independent\\
  \item
    \(C_\tau \to W\) with \(H(t) = 2t - t^2\)
  \end{itemize}
\end{itemize}
\end{frame}

\begin{frame}{Results I}
\phantomsection\label{results-i}
Under various settings, we compare the following methods:

\begin{itemize}
\item
  \(S_r\): Pearson--correlation \(L_2\)-type test; best for \emph{dense
  dependence}.
\item
  \(TS_{\tau}\): Kendall's tau \(L_2\)-type test using \emph{asymptotic}
  critical values.
\item
  \(MS_{\tau}\): Same as \(TS_{\tau}\) but with \emph{Monte Carlo}
  critical values.
\item
  \(M_r\): Pearson--correlation \(L_{\infty}\)-type test; best for
  \emph{sparse dependence}.
\item
  \(TM_{\tau}\): Kendall's tau \(L_{\infty}\)-type test using
  \emph{asymptotic} Gumbel limits.
\item
  \(MM_{\tau}\): Same as \(TM_{\tau}\) but with \emph{Monte Carlo}
  critical values.
\item
  \(TC_{\tau}\): \emph{Adaptive} Kendall's tau test combining
  \(S_{\tau}\) and \(M_{\tau}\) (asymptotic).
\item
  \(MC_{\tau}\): \emph{Adaptive} Kendall's tau test combining
  \(S_{\tau}\) and \(M_{\tau}\) (Monte Carlo).
\item
  \(PE_r\): \emph{Power-enhanced} Pearson test improving \(S_r\) under
  sparse cases.
\item
  \(U_{\min}\): \emph{Adaptive U-statistic} test combining multiple
  orders via minimum p-value.
\end{itemize}
\end{frame}

\begin{frame}{Results II}
\phantomsection\label{results-ii}
There are a number of tables for the various conditions the authors
evaluated the statistical tests. The main focus is on the \emph{Size}
and \emph{Power} (mainly the latter) of the statistical tests.

Overall:

\begin{itemize}
\tightlist
\item
  While one particular non-adaptive test may do best under a particular
  setting (dense/sparse),
\item
  Both implementations of the adaptive test perform about as well as the
  ``best'' method in each scenario, but across \textbf{both} dense and
  sparse settings.
\item
  Highlighted portions of the tables that follow are the adaptive tests
\end{itemize}
\end{frame}

\begin{frame}{Results III}
\phantomsection\label{results-iii}
\begin{figure}

{\centering \includegraphics[width=0.8\linewidth,height=0.77\textheight]{Figures/Table1H} 

}

\caption{Empirical sizes of tests (small is good)}\label{fig:Table 1}
\end{figure}
\end{frame}

\begin{frame}{Results IV}
\phantomsection\label{results-iv}
\begin{figure}

{\centering \includegraphics[width=0.8\linewidth]{Figures/Table2H} 

}

\caption{Empirical powers of tests in dense cases.}\label{fig:Table 2}
\end{figure}
\end{frame}

\begin{frame}{Results V}
\phantomsection\label{results-v}
\begin{figure}

{\centering \includegraphics[width=0.8\linewidth]{Figures/Table3H} 

}

\caption{Empirical powers of tests in sparse cases.}\label{fig:Table 3}
\end{figure}
\end{frame}

\begin{frame}{Results VI}
\phantomsection\label{results-vi}
\begin{figure}

{\centering \includegraphics[width=0.8\linewidth,height=0.77\textheight]{Figures/Table4H} 

}

\caption{Empirical powers under various strengths of dependence in dense cases.}\label{fig:Table 4}
\end{figure}
\end{frame}

\begin{frame}{Results VII}
\phantomsection\label{results-vii}
\begin{figure}

{\centering \includegraphics[width=0.8\linewidth]{Figures/Table5H} 

}

\caption{Empirical powers under various strengths of dependence in sparse cases.}\label{fig:Table 5}
\end{figure}
\end{frame}

\begin{frame}{Applications}
\phantomsection\label{applications}
\begin{itemize}
\tightlist
\item
  The paper also applies the method to two ``real-world'' examples:

  \begin{itemize}
  \tightlist
  \item
    \textbf{Welding (4 vars, n=40):} Rank-based rejects null hypothesis;
    Pearson fails to reject.
  \item
    \textbf{Biochemical (8 vars, n=12):} Adaptive test detects group
    differences.
  \end{itemize}
\item
  I also implemented the method myself (and validated it against the
  authors' implementation).

  \begin{itemize}
  \tightlist
  \item
    Applied (mainly for fun) to a Consulting Case
  \item
    A priori believed the covariate to be independent, and they were (at
    least according to the adaptive test)!
  \end{itemize}
\end{itemize}
\end{frame}

\begin{frame}{Conclusion}
\phantomsection\label{conclusion}
\begin{itemize}
\tightlist
\item
  Kendall's \(\tau\) connects classic nonparametric tests to modern
  high-dimension inference.
\item
  Rank-based adaptive tests are practical and robust; 2025 work
  generalizes to sum-of-powers (Han, Ma, and Xie 2025).
\item
  We should consider this test when we know we have high-dimensional
  data, but don't know whether it is dense or sparse.
\item
  Allows us to test for independence beyond the typical ``data
  collection assessment'' and ``variable relations graphs''
\end{itemize}
\end{frame}

\begin{frame}[allowframebreaks]{References}
\phantomsection\label{references}
\scriptsize

\phantomsection\label{refs}
\begin{CSLReferences}{1}{0}
\bibitem[\citeproctext]{ref-ElShaarawi1992}
El-Shaarawi, A. H., and Stefan P. Niculescu. 1992. {``On Kendall's Tau
as a Test of Trend in Time Series Data.''} \emph{Environmetrics} 3 (4):
385--411.

\bibitem[\citeproctext]{ref-Fechner1897}
Fechner, Gustav Theodor. 1897. \emph{Kollektivmasslehre}. Leipzig:
Verlag von Wilhelm Engelmann.
\url{https://www.google.com/books/edition/Kollektivmasslehre/bgQZAAAAMAAJ?hl=en}.

\bibitem[\citeproctext]{ref-Hamed2011}
Hamed, K. H. 2011. {``The Distribution of Kendall's Tau for Testing the
Significance of Cross-Correlation in Persistent Data.''}
\emph{Hydrological Sciences Journal} 56 (5): 841--53.
\url{https://doi.org/10.1080/02626667.2011.586948}.

\bibitem[\citeproctext]{ref-Han2025}
Han, Lijuan, Yun Ma, and Junshan Xie. 2025. {``An Adaptive Test of the
Independence of High-Dimensional Data Based on Kendall Rank Correlation
Coefficient.''} \emph{Journal of Nonparametric Statistics} 37 (3):
632--56. \url{https://doi.org/10.1080/10485252.2024.2435852}.

\bibitem[\citeproctext]{ref-Kendall1938}
Kendall, M. G. 1938. {``A New Measure of Rank Correlation.''}
\emph{Biometrika} 30 (1/2): 81--93.
\url{https://www.jstor.org/stable/2332226}.

\bibitem[\citeproctext]{ref-Kruskal1958}
Kruskal, William H. 1958. {``Ordinal Measures of Association.''}
\emph{Journal of the American Statistical Association} 53 (284):
814--61. \url{https://doi.org/10.1080/01621459.1958.10501481}.

\bibitem[\citeproctext]{ref-Shi2024}
Shi, Xiangyu, Yuanyuan Jiang, Jiang Du, and Zhuqing Miao. 2024. {``An
Adaptive Test Based on Kendall's Tau for Independence in High
Dimensions.''} \emph{Journal of Nonparametric Statistics} 36 (4):
1064--87. \url{https://doi.org/10.1080/10485252.2023.2296521}.

\end{CSLReferences}
\end{frame}

\end{document}
