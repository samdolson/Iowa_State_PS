\documentclass[11pt]{article}
\usepackage{graphicx}
\usepackage{amsmath,latexsym,amssymb}
\usepackage{color}
\setlength{\oddsidemargin}{-0.25in}
\setlength{\textwidth}{7.1in}
\setlength{\topmargin}{-1in}
\setlength{\textheight}{9.5in}
\evensidemargin \oddsidemargin
\newcommand{\red}[1]{{\color{red} #1}}
\newcommand{\blue}[1]{{\color{blue} #1}}
\usepackage{url}
\usepackage{parskip}
\usepackage{enumitem}

\begin{document}

\large \noindent \textsc{Stat 5000}\hfill \large{\textsc{Lab \#2}} \hfill \phantom{\textsc{Name:} \underline{\hspace{2in}}} \\
\large \textsc{Fall 2024}  \hfill \textsc{Due Tue Sep 10th} \hfill \textsc{Name:} \underline{\hspace{2in}}

\vspace{14pt}

\noindent \textbf{Directions:} Complete the exercises below. When you are finished, turn in any required files online in Canvas, then check-in with the Lab TA for dismissal.
\\ \underline{\hspace{7in}}
\vspace{14pt}

\textbf{\underline{Introduction to t-Tests in R}}

Refer to the \texttt{fuel\_economy.csv} data file posted in Canvas. This data set has information about an observational study of automobiles driven in Canada, including the following two columns: 
\begin{description}
	\item [\texttt{Cylinders}:] category variable with two levels - 4 or 6
	\item [\texttt{Consumption}:] numeric response variable with the fuel consumption in miles per gallon (mpg)
\end{description}
Researchers are interested in exploring whether there is a difference in the average fuel consumption of vehicles with engines built using differing numbers of cylinders. The code to conduct a two-sample t-test in R is explained below. The full R program is provided in the file \texttt{fuel\_economy\_Lab2.R} posted on Canvas.

\begin{itemize}
	\item First, load in the data using the \textit{Import Dataset} tool in R Studio. Be sure to change the variable type on the Cylinders column to ``factor" and enter ``4, 6" as the levels.
	\begin{verbatim}
		library(readr)
		fuel <- read_csv("fuel_economy.csv", 
		                      col_types=cols(Cylinders=col_factor(levels=c("4", "6"))))
		View(fuel)
	\end{verbatim}
	\item Next, compute the corresponding summary statistics within in group.
	\begin{verbatim}
		library(tidyverse)
		sum_stats = fuel |> 
  			group_by(Cylinders) |>
  			summarize(
    			Y_n = n(),
    			Y_mean = mean(Consumption.mpg),
    			Y_sd = sd(Consumption.mpg)
  			)
		sum_stats
	\end{verbatim}
	\item Then, use the \texttt{t.test()} function to conduct a test for the difference in mean fuel consumption between 4 and 6 cylinder vehicles. Indicate the response variable name before the $\sim$ and the category variable name after, use the \texttt{data} option to provide the name of the dataset, and use the \texttt{var.equal} option set to ``TRUE" to indicate the population variances are assumed equal.
	\begin{verbatim}
		HT = t.test(Consumption.mpg~Cylinders, data=fuel, var.equal=TRUE)
		HT
	\end{verbatim}
	You can see what pieces of information are stored in the \texttt{HT} variable using the \texttt{names()} function. You can access these pieces of information using the \texttt{\$} operator, e.g.
	\begin{verbatim}
		names(HT)
		HT$null.value
	\end{verbatim}
\end{itemize}

\newpage

\textbf{\underline{Assignment}}
\begin{enumerate}
	\item State the hypotheses for the two-sided test.
	\item From the output, find/compute the difference in the two sample means.
	\item From the output, find/compute the estimate of the pooled standard deviation.
	\item From the output, find/compute the test statistic for the hypothesis test. 
	\item From the output, find/compute the degrees of freedom for the test. 
	\item From the output, find/compute the $p$-value for the two-sided hypothesis test.
	\item Interpret the results of the two-sided test in the context of the research question.
	\item By default, R conducts the two-sided hypothesis test. You can change this by adding the parameter ``\texttt{alternative=greater}'' or ``\texttt{alternative=less}'' inside the \texttt{t.test()} function. Provide a research question corresponding to either the ``greater" or ``less" one-sided test. 
\end{enumerate}

\vfill
\textbf{Total:} 25 points \hspace{14pt} \textbf{\# correct:} \underline{\hspace{1in}}  \hspace{14pt} \textbf{\%:} \underline{\hspace{1in}} 

\end{document}








